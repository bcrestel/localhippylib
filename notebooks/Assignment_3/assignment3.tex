
\documentclass[12pt]{article}

%% \usepackage{times}
% \usepackage{newcent}
%% FONTS
%% To get the default sans serif font in latex, uncomment following line:
\renewcommand*\familydefault{\sfdefault}
%%
%% to get Arial font as the sans serif font, uncomment following line:
%% \renewcommand{\sfdefault}{phv} % phv is the Arial font
%%
%% to get Helvetica font as the sans serif font, uncomment following line:
% \usepackage{helvet}
\usepackage{wrapfig}
%% NATBIB is the one to use here. Just be careful since it puts space
%% between references; there's probably some customization that
%% compresses the space out (like the cite package does).
%% For CDI, space was tight so we went back to cite
% \usepackage[square,numbers,sort&compress]{natbib}
%% note that hyperref has problems with the cite package--better to
%% use natbib
\usepackage[sort,nocompress]{cite}
% \usepackage[margin=0pt,font=small,labelsep=space,labelfont=bf]{caption}
\usepackage[small,bf,up]{caption}
\renewcommand{\captionfont}{\footnotesize}
\usepackage[left=1in,right=1in,top=1in,bottom=1in]{geometry}
\usepackage{graphics,epsfig,graphicx,float,subfigure,color}
\usepackage{algorithm,algorithmic}
\usepackage{amsmath,amssymb,amsbsy,amsfonts,amsthm}
% \usepackage{subsec}
\usepackage{comment}
\usepackage{url}
\usepackage{boxedminipage}
\usepackage[sf,bf,small]{titlesec}
% \usepackage[textsize=footnotesize]{todonotes}
 \usepackage[plainpages=false, colorlinks=true,
   citecolor=blue, filecolor=blue, linkcolor=blue,
   urlcolor=blue]{hyperref}
% \usepackage{url}

\newcommand{\todo}[1]{\textcolor{red}{#1}}
% see documentation for titlesec package
% \titleformat{\section}{\large \sffamily \bfseries}
\titlelabel{\thetitle.\,\,\,}

% \renewcommand{\baselinestretch}{0.994}

\newcommand{\gbf}[1]{\text{\boldmath${#1}$\unboldmath}}
\newcommand{\bs}{\boldsymbol}
\newcommand{\Div}{\text{div}}

\newcommand{\edot}{\dot{\gbf{\varepsilon}}}
\newcommand{\secinve}{\edot_\mathrm{II}}
\newcommand{\secinvt}{\gbf{\tau}_\mathrm{II}}
\newcommand{\cD}{\mathcal{D}}
\newcommand{\cF}{\mathcal{F}}
\newcommand{\cL}{\mathcal{L}}
\newcommand{\cR}{\mathcal{R}}

\newcommand{\obs}{\mathrm{obs}}

\newcommand{\bit}{\begin{itemize}}
\newcommand{\eit}{\end{itemize}}

\newcommand{\bdm}{\begin{displaymath}}
\newcommand{\edm}{\end{displaymath}}

\newcommand{\ben}{\begin{enumerate}}
\newcommand{\een}{\end{enumerate}}

\newcommand{\alert}[1]{\textcolor{red}{#1}}
% \newcommand{\alert}[1]{\textcolor{SkyBlue3}{#1}}

\newcommand{\mcone}[1]{\multicolumn{1}{c}{#1}}
\newcommand{\tentwo}[1] {\ensuremath{\boldsymbol{#1}}}
\newcommand{\tenfour}[1] {\ensuremath{\boldsymbol{\mathsf{#1}}}}
\renewcommand{\vec}[1] {\ensuremath{\boldsymbol{#1}}}

\newcommand{\zapspace}{\topsep=0pt\partopsep=0pt\itemsep=0pt\parskip=0pt}

\newcommand{\footnoteremember}[2]{\footnote{#2}\newcounter{#1}\setcounter{#1}{\value{footnote}}}

\newcommand{\footnoterecall}[1]{\footnotemark[\value{#1}]}

\setlength{\emergencystretch}{20pt}

\begin{document}

\begin{center}
\large \textsf{\textbf{%%
Spring 2014: \\ 
Computational and Variational Methods for Inverse Problems\\
CSE 397/GEO 391/ME 397/ORI 397\\
Assignment 3 (due 24 March 2014)}}
\end{center}

\noindent
The problems below require a mix of paper-and-pencil work and FEniCS implementation.
Example implementations in FEniCS (as used for Problems 2 and 3 below) can be found on the class
website,
\url{http://users.ices.utexas.edu/~omar/inverse_probs/index.html}. Please
hand in printouts of your FEniCS implementations together with the
results.

\begin{enumerate}

\item The problem of removing noise from an image without blurring
sharp edges can be formulated as an infinite-dimensional minimization
problem. Given a possibly noisy image $u_0(x,y)$ defined within a
square domain $\Omega$, we would like to
find the image $u(x,y)$ that is closest in the $L_2$ sense, i.e.\ we
want to minimize 
%
\bdm
\mathcal{F}_{LS} := \frac{1}{2}\int_\Omega (u - u_0)^2 \; d\boldsymbol{x},
\edm
%
while also removing noise, which is assumed to comprise very ``rough''
components of the image. This latter goal can be incorporated as an
additional term in the objective, in the form of a penalty, i.e., 
%
\bdm \mathcal{R}_{TN} := \! \frac{1}{2}\int_\Omega k(\boldsymbol{x}) \nabla u
\cdot \! \nabla u \; d\boldsymbol{x}, 
\edm 
%
where $k(\boldsymbol{x})$ acts as a ``diffusion'' coefficient that controls
how strongly we impose the penalty, i.e.\ how much smoothing
occurs. Unfortunately, if there are sharp edges in the image, this
so-called {\em Tikhonov (TN) regularization} will blur them. Instead,
in these cases we prefer the so-called {\em total variation (TV)
regularization},
%
\bdm
\mathcal{R}_{TV} := \! \int_\Omega k(\boldsymbol{x}) (\nabla u \cdot \! \nabla
u)^{\frac{1}{2}} \; d\boldsymbol{x}
\edm
%
where (we will see that) taking the square root is the key to
preserving edges. Since 
$\mathcal{R}_{TV}$ is not differentiable when $\nabla u =
\boldsymbol{0}$, it is usually modified to include a positive parameter
$\varepsilon$ as follows:
\bdm
\mathcal{R}^{\varepsilon}_{TV} := \!  \int_\Omega k(\boldsymbol{x}) (\nabla u \cdot
\! \nabla u + \varepsilon)^{\frac{1}{2}} \; d\boldsymbol{x}.
\edm
We wish to study the performance of the two denoising functionals
$\mathcal{F}_{TN}$ and $\mathcal{F}^{\varepsilon}_{TV}$, where
\bdm
\mathcal{F}_{TN}  := \mathcal{F}_{LS} + \mathcal{R}_{TN}  
\edm
and
\bdm
\mathcal{F}^{\varepsilon}_{TV}  := \mathcal{F}_{LS} +
      \mathcal{R}^{\varepsilon}_{TV}.
\edm
%
We will prescribe the homogeneous Neumann condition 
$\nabla u \cdot \boldsymbol{n}=0$ on the four sides of the square,
which amounts to assuming that the image intensity does not change
normal to the boundary of the image. 

\ben

\item For both $\mathcal{F}_{TN}$ and
$\mathcal{F}^{\varepsilon}_{TV}$, derive the 
first-order necessary condition for optimality using calculus of
variations, in both weak form and strong form. Use $\hat{u}$ to represent
the variation of $u$.

\item Show that when $\boldsymbol{\nabla}u$ is zero, $\mathcal{R}_{TV}$ is not
differentiable, but $\mathcal{R}^{\varepsilon}_{TV}$ is.

\item For both $\mathcal{F}_{TN}$ and
  $\mathcal{F}^{\varepsilon}_{TV}$, derive the infinite-dimensional
  Newton step, in both weak and strong form. For consistency of
  notation, please use $\tilde{u}$ as the differential of $u$
  (i.e.\ the Newton step). The strong form of the second variation of
  $\mathcal{F}^{\varepsilon}_{TV}$ will give an anisotropic diffusion
  operator of the form $-\Div(\boldsymbol{A}(u) \nabla \tilde{u})$,
  where $\boldsymbol{A}(u)$ is an anisotropic tensor that plays the
  role of the diffusivity coefficient\footnote{Hint: For vectors
    $a,b,c\in \mathbb{R}^n$, note the identity $(a\cdot b)c =
    (ca^T)b$, where $a\cdot b\in \mathbb{R}$ is the inner product and
    $ca^T\in \mathbb{R}^{n\times n}$ is a matrix of rank one.}. (In
  contrast, you can think of the second variation of
  $\mathcal{F}_{TN}$ giving an {\em isotropic} diffusion operator,
  i.e.\ with $\boldsymbol{A}= \alpha \boldsymbol{I}$ for some
  $\alpha$.)

\item Derive expressions for the two eigenvalues and corresponding
  eigenvectors of $\boldsymbol{A}$. Based on these expressions, give
  an explanation of why $\mathcal{F}_{TV}^{\varepsilon}$ is effective
  at preserving sharp edges in the image, while $\mathcal{F}_{TN}$ is
  not. Consider a single Newton step for this argument.

\item Show that for large
enough $\varepsilon$, $\mathcal{R}^{\varepsilon}_{TV}$ behaves like
$\mathcal{R}_{TN}$, and for $\varepsilon=0$, the Hessian of
$\mathcal{R}^{\varepsilon}_{TV}$ is singular. This suggests that
$\varepsilon$ should be chosen small enough that edge preservation is
not lost, but not too small that ill-conditioning occurs.

%\item Show the equivalence between the following two approaches:
%\bit
%\item {\em Ritz method (i.e.\ ``discretize-then-optimize'')}: First make a finite element
%approximation of the infinite-dimensional functional
%$\mathcal{F}^{\varepsilon}_{TV}$, and then derive the
%finite-dimensional Newton step.
%%
%\item {\em Galerkin method (i.e.\ ``optimize-then-discretize'')}: Directly make a finite
%element approximation of the (weak form of the) infinite-dimensional
%Newton step you derived above.
%\eit

\een

%\end{enumerate}
% ****************************

\item An anisotropic Poisson problem in a two-dimensional domain
  $\Omega$ is given by the strong form
\begin{subequations}\label{eq:P}
\begin{align}\label{eq:P1}
-\nabla \cdot\left( \bs{A} \nabla u\right) &= f \quad
\:\:\text{ in }\Omega, \\ 
u &= u_0  \quad \text{ on }\partial\Omega,
\end{align}
\end{subequations}
where the conductivity tensor $\bs{A}(\bs{x})\in \mathbb{R}^{2\times
  2}$ is assumed to be symmetric and positive definite for all
$\bs{x}$, $f(\bs{x})$ is a given distributed source, and $u_0(\bs{x})$
is the boundary source.\footnote{One interpretation of
  Eqn.\ \eqref{eq:P} is that it describes the steady state conduction
  of heat in a solid body. In this case, $u$ is the temperature,
  $\bs{A}$ is the thermal conductivity, $f$
  is the distributed heat source, and the temperature at the boundary
  $\partial\Omega$ is maintained at $u_0$.}

\begin{enumerate}
\item Derive the variational/weak form corresponding to the above
    problem, and give the energy functional that is minimized by the
  solution $u$ of \eqref{eq:P}.
\item Solve problem \eqref{eq:P} in FEniCS using quadratic finite
  elements. Choose $\Omega$ to be a disc with radius 1 around the
  origin and take the source terms to be
\begin{equation*}
f = \exp(-100(x^2+y^2))\quad \text{ and } \quad u_0 = 0.
\end{equation*}
Use conductivity tensors $A(x)$ given by
\begin{equation*}
A_1 = \begin{pmatrix}
10 & 0\\
0  &10
\end{pmatrix}
\text{ and }
A_2 = \begin{pmatrix}
1  & -5\\
-5 &100
\end{pmatrix}
\end{equation*}
and compare the results obtained using $A_1$ and $A_2$ in \eqref{eq:P}.

{\bf HINTS:}
\begin{itemize}
\item [-] A mesh for the unit circle is provided in the file \verb+circle.xml+. You can load the mesh in FEniCS using the command
\begin{verbatim}
mesh = Mesh("circle.xml")
\end{verbatim}
\item [-] To define the conductivity tensor ${\bs A}({\bs x})$ use the commands
\begin{verbatim}
A1 = Expression((("10", "0"),("0", "10")))
A2 = Expression((("1", "-5"),("-5", "100")))
\end{verbatim}
\end{itemize}

\end{enumerate}

% ****************************

\item
Implement the image denoising method from Problem 1 above using
Tikhonov (TN) and total variation (TV) regularizations. To this end,
set $k(\boldsymbol{x})=\alpha$ with small $\alpha>0$ in
$\mathcal{R}_{TV}$ and $\mathcal{R}_{TN}$, choose small $\varepsilon
>0$, and take a homogeneous Neumann boundary condition for $u$ (i.e.,
$\nabla u \cdot \bs{n} = 0$).  A function that evaluates the noisy
image $u_0(\bs{x})$ is provided in the file \verb+ustar_fun.m+, while
the file \verb+TNTV.m+ contains some lines to start up the
implementation.
\begin{enumerate}

\item Solve the denoising inverse problem using TN
  regularization. Since for TN regularization, the gradient is linear
  in $u$, you can use COMSOL's linear solver \verb+femlin+. Choose an
  $\alpha>0$ such that you obtain a reasonable
  reconstruction\footnote{Either experiment manually with a few values
    for $\alpha$ or use the L-curve criterion. To evaluate terms in
    the cost functional, use the function {\ttfamily postint}.}, i.e.,
  a reconstruction that removes noise from the image but  does not
  overly smooth the image.

\item Solve the denoising inverse problem using TV
  regularization. Since the gradient is nonlinear in $u$, use the
  nonlinear solver \verb+femnlin+\footnote{This function implements a
    Newton method with damping to solve the weak form of a nonlinear
    equation, using symbolic variations of the residual to obtain the
    Jacobian for Newton's method. While COMSOL's nonlinear solver will
    work for this problem (and you should use it), it does not work
    for every nonlinear system of PDEs, which is why it is often
    necessary to derive and implement Newton's method manually (via
    explicit solution of a sequence of linear problems).  One
    situation where explicit implementation of Newton's method is
    useful is when the nonlinear equation corresponds to the vanishing
    of the gradient of an optimization problem (as it is in the
    denoising problem). In this case, COMSOL does not know the
    underlying optimization cost functional, so manual implementation
    allows one to globalize Newton's method via a line search based on
    a knowledge of the cost functional.}. Find an appropriate value
  for $\alpha$\footnote{Try out a few different values for
    $\alpha$. Using the L-curve or Morozov's criterion in not
    justified here, since compared to TN regularization, TV
    regularization is not quadratic, which makes the automatic choice
    of $\alpha$ harder.}. You will have to increase the default number
  of nonlinear iterations in \verb+femnlin+\footnote{Typing
    ``{\ttfamily help femnlin}'' will show you solver options. You
    should increase the value of {\ttfamily Maxiter}, which defaults
    to 25.}. How does the number of nonlinear iterations behave for
  decreasing $\varepsilon$ (e.g., between $10$ and $10^{-4}$)? Try to
  explain this behavior\footnote{There are more efficient so-called
    primal-dual Newton algorithms for TV-regularized problems (see,
    for instance, {\em T.F.\ Chan, G.H.\ Golub, and P. Mulet,
      A nonlinear primal-dual method for total variation-based image
      restoration, SIAM Journal on Scientific Computing,
      20(6):1964--1977, 1999}). The efficient solution of
    TV-regularized problems is still an active field of research.}.

\item Compare the denoised images obtained with TN and TV
  regularizations, using the insight derived from your answers to
  Problem 1.

\end{enumerate}

% **********************
%\item
%We solve the following inverse problem for a distributed source $g \in
%L_2(\Omega)$ on a domain $\Omega$ with boundary $\partial\Omega$.
%\begin{subequations}\label{eq:optcon}
%\begin{align}\label{eq:optcon1}
%\min_{g} J(g) := \frac 12 \int_\Omega (u-u^{obs})^2\,dx + \frac{\alpha}{2}\int_\Omega g^2\,dx,
%\end{align}
%where $u=u(g)$ is given by the solution of
%\begin{align}
%-\nabla\cdot(k(x)\nabla u) &= g \:\:\text{ in } \Omega,\label{eq:optcon2}\\
%u &= 0  \:\:\text{ on } \partial\Omega,\label{eq:optcon3}
%\end{align}
%\end{subequations}
%where $k(x)>0$ is given and $\alpha>0$ is the regularization parameter.
%\begin{enumerate}
%\item Give the Lagrangian functional using the weak form of
%  \eqref{eq:optcon2} and \eqref{eq:optcon3}.
%\item Derive the gradient of $J$ with respect to $g$. Give both the
%  strong as well as the weak forms for state, the adjoint and the
%  gradient equation.
%\item Since the state, the adjoint and the gradient equation are
%  linear, the solution of \eqref{eq:optcon} can be obtained by solving
%  a large linear system\footnote{This system can be understood as
%    Newton step---for linear problems, Newton's method converges in a
%    single step.}. Use COMSOL to solve the problem for
%  $\Omega=[0,1]\times[0,1]$, $u^{obs}\equiv 1$ and $k(x)\equiv 1$:
%  introduce finite element basis functions for $u$, $p$ and $g$, sum
%  the weak forms for the state, adjoint and gradient equation, and use
%  the linear solver \verb+femlin+ to solve the linear system. Plot the
%  solution $u$ for a series of decreasing values of regularization
%  parameters $\alpha$ and discuss the results. Can $u$ be equal to
%  $u^{obs}$, which would make the misfit term zero?
%\end{enumerate}
%

\end{enumerate}





%% \item The image registration problem is to find a mapping of a
%%   ``patient'' image
%% of intensity $\mathcal{I}_p(\boldsymbol{x})$ to a ``template'' image of intensity
%% $\mathcal{I}_t(\boldsymbol{x})$. This can be formulated as finding the displacement
%% vector $\boldsymbol{u}$ that minimizes the least squares error functional 
%% \bdm
%% \mathcal{F} := \frac{1}{2} \int_\Omega \! 
%% [\mathcal{I}_t(\boldsymbol{x}) - \mathcal{I}_p(\boldsymbol{x}+ \boldsymbol{u})]^2 \;
%% \boldsymbol{dx} 
%% % + \frac{\alpha}{2} \int_\Omega \! \left[\frac{\lambda}{\mu} |\nabla \gbf{u}
%% % + \nabla \gbf{u}^T|^2 + |\Div \gbf{u}|^2 \right] \; \gbf{dx}
%% \edm
%% \ben
%% \item Derive the first order optimality condition in weak and strong
%%   form. Show that for piecewise-homogeneous images, the optimality
%%   condition implies that it is only the edge regions of the (patient)
%%   image that determine the registration.  (Use $\hat{\boldsymbol{u}}$
%%   to denote the variation of $\boldsymbol{u}$.)
%% 
%% \item Derive an expression for the Newton step   $\tilde{\boldsymbol{u}}$ in
%%   both weak and strong form. Show that the Hessian contains
%%   information only in directions normal to image contours, in
%%   situations when the misfit between images, i.e.\ $\mathcal{I}_t(\boldsymbol{x}) -
%%   \mathcal{I}_p(\boldsymbol{x}+ \boldsymbol{u})$, is negligible (such as near an
%%   optimum). Thus, the Hessian is singular with respect to tangential
%%   directions, and the problem of minimizing $\mathcal{F}$ is
%%   ill-posed. 
%% 
%% \item Suggest a regularization term that can be added to $\mathcal{F}$
%%   to render the problem
%%   well-posed. 
%% \een
%\een


\end{document}







\item Consider the unconstrained optimization problem
\bdm
\min f(x,y) \equiv - \cos x \cos y/10.
\edm
\ben
\item Find and classify all stationary points in the region 
$-\pi/2 \leq x \leq \pi/2, -10\pi/2 \leq y \leq 10\pi/2$
% $-\pi/2 \leq x,y \leq \pi/2$.
\item There is a portion of the problem region  
% $-\pi/2 \leq x,y \leq \pi/2$
within which the Hessian matrix of $f(x,y)$ is positive definite.
Give expressions for this portion. You should be able to do this
analytically.
\item Derive expressions for the search directions associated with the
steepest descent and Newton methods. 
\item Write a program that performs both iterations, both without a
  line search and with an exact line search.  Note that you will not
  be able to find the value of the optimal step length analytically;
  instead, determine it numerically. {\em Suggestion:} use a built-in
  one-dimensional minimization function such as {\tt FindMinimum} or
  {\tt FindRoot} in {\sc Mathematica} or {\tt fzero} in {\sc Matlab}.
\item Run your program for various initial guesses within the region.
% $-\pi/2 \leq x,y \leq \pi/2$. 
Verify the following:
\ben
\item Steepest descent converges to the minimum $x^*$ for any starting
point within the region.
\item Newton's method with line search converges to the minimum only
for initial points for which the Hessian matrix is positive definite.
\item Newton's method without line search has an even more restricted
radius of convergence. {\em Optional:} Determine that radius of
convergence.
\een
\item What do you observe about the convergence rate in these cases?
{\em Optional:} Verify the observed convergence rate analytically.

\een
% \item Repeat the above for the problem
% \bdm
% \min f(x,y) \equiv - \cos x \cos (y/3)
% \edm
% and the region $-\pi/2 \leq x \leq \pi/2, -3\pi/2 \leq y \leq
% 3\pi/2$.
% \een

% defer to linear constraints homework

\newpage

\item Write a program that implements the inexact Newton-conjugate
  gradient method as described in class\footnote{Reference:
    S.C. Eisenstat and H.F Walker, {\em Globally convergent inexact
      Newton's method}, SIAM Journal on Optimization, Vol.\ 4,
    p.393--422, 1994.}  and use it to solve the following problem
  (known as the multidimensional Rosenbrock's function):
%
\bdm
\min f(x_1, x_2, \ldots, x_n) := \sum_{i=1}^{n/2} \left[ 100 (x_{2i-1}^2 - x_{2i})^2
  + (x_{2i-1} - 1)^2 \right]
\edm
%  \bdm \min f(x,y) \equiv 100(y - x^2)^2 + (1-x)^2 \edm 
Your implementation should have the following features:
%
\bit
\item Terminate the CG iterations when $|| H_k p_k + g_k || \le \eta
  ||g_k|| $, and implement the following three options for $\eta $:
\bit
\item $\eta_k = 0.5$
\item $\eta_k = \min(0.5, \sqrt{||g_k||})$
\item $\eta_k = \min(0.5, ||g_k||)$
\eit
%
\item Also terminate the CG iterations when a direction of negative
  curvature is detected within the CG iteration.

\item For a line search, implement a backtracking Armijo line search
  as described in class. 

\eit
\ben

\item Compare the performance of Newton and steepest descent for $n=2$
  (i.e.\ in two dimensions). Plot the sequence of iterates for each
  method on top of the contour plot of $f(\bs{x})$. 

\item Experiment with the different choices of $\eta$ 
% and with different initial guesses, 
for $n=100$. Verify the theoretical convergence rates for
the different choices of $\eta$.  

\item What do you observe about the number of iterations taken by
  Newton as the problem dimension increases from $n=10$ to $n=100$ to
  $n=1000$? [use the consistent starting point ($-1.2, 1.0, -1.2, 1.0,
  -1.2, 1.0$, etc.)].

\een
\een

\end{document}



