%% NSF allows 10pt Arial, but that violates the reviewers' 8th
%% amendment rights 
\documentclass[11pt]{article}

%\usepackage{times}
% \usepackage{newcent}
%% FONTS
%% To get the default sans serif font in latex, uncomment following line:
\renewcommand*\familydefault{\sfdefault}
%%
%% to get Arial font as the sans serif font, uncomment following line:
%\renewcommand{\sfdefault}{phv} % phv is the Arial font
%%
%% to get Helvetica font as the sans serif font, uncomment following line:
%% \usepackage{helvet}
\usepackage{wrapfig}

%% the natbib package works better than cite 
\usepackage[square,numbers,sort&compress]{natbib}

\usepackage[small,bf,up]{caption}
\renewcommand{\captionfont}{\footnotesize}
\usepackage[left=1in,right=1in,top=1in,bottom=1in]{geometry}
\usepackage{graphics,epsfig,graphicx,float,subfigure,color}
\usepackage{algorithm,algorithmic}
\usepackage{amsmath,amssymb,amsbsy,amsfonts,amsthm}
%% \usepackage{subsec}
\usepackage{comment}
\usepackage{url}
\usepackage{boxedminipage}
\usepackage[sf,bf,small]{titlesec}
%% \usepackage[textsize=footnotesize]{todonotes}

% see documentation for titlesec package
% \titleformat{\section}{\large \sffamily \bfseries}
\titlelabel{\thetitle.\,\,\,}

%% as a last resort, if we're short on space: 
%% \renewcommand{\baselinestretch}{0.99}
\newcommand{\bit}{\begin{itemize}}
\newcommand{\eit}{\end{itemize}}

\newcommand{\zapspace}{\topsep=0pt\partopsep=0pt\itemsep=0pt\parskip=0pt}
\newcommand{\todo}[1]{\textcolor{red}{#1}}

% \addtolength{\oddsidemargin}{-1.0in}%{-0.75in}
% \addtolength{\textwidth}{1.75in}
% \addtolength{\headsep}{-0.75in}
% \addtolength{\textheight}{2.0in}

% \setlength{\textheight}{9.0in}
% \setlength{\textwidth}{6.5in}
% \setlength{\parskip}{1ex}
% \setlength{\parindent}{0em}

\setlength{\emergencystretch}{20pt}

\usepackage{amsmath,amssymb,amsbsy,amsfonts,amsthm,mathrsfs}
\usepackage{fullpage,subfigure,graphicx,epsfig,url,color}
\usepackage[plainpages=false, colorlinks=true,
   citecolor=black, filecolor=black, linkcolor=black,
   urlcolor=blue]{hyperref}

%\include{ogmacros}

\newcommand{\bdm}{\begin{displaymath}}
\newcommand{\edm}{\end{displaymath}}

\newcommand{\ben}{\begin{enumerate}}
\newcommand{\een}{\end{enumerate}}

\newcommand{\p}{\partial}
\newcommand{\bs}{\boldsymbol}

\renewcommand{\matrix}[1]{\ensuremath{\boldsymbol{#1}}}
\newcommand{\K}{\ensuremath{\matrix{K}}}
\newcommand{\F}{\ensuremath{\matrix{F}}}



\begin{document}
\pagestyle{empty}
\begin{center}
% {\large {\bf GEO 384F: Computational Methods for Geophysics}}\\
{\large {\bf Getting started with FEniCS}}\\
\end{center}

\section{FEniCS overview}

FEniCS is a powerful, open-source suite of tools for automated solution
of PDEs using finite elements. Part of the power for FEniCS is the
ease with which one can create FE solvers by describing PDEs using
weak forms in nearly-mathematical notation. The FEniCS software, in
addition to extensive documentation and examples, can be found at the
FEniCS Project website, \url{http://fenicsproject.org/}. 

FEniCS includes a number of powerful features, which are described at
\url{http://fenicsproject.org/about/features.html}. These include: 
\begin{itemize}
\zapspace
\item Automated solution of variational problems

\item Automated error control and adaptivity

\item An extensive library of finite elements

\item High performance linear algebra through backends to such libraries
  as PETSc and Trilinos. 

\item Visualization via a simple interactive plotting function, as
  well as output in VTK format 
% for visualization in ParaView. 

\item FEniCS can be used from both Python and C++

\item Extensive documentation

\end{itemize}

\section{FEniCS resources}

The documentation for FEniCS is extensive. Resources include:
\bit
 \zapspace
 
 \item {\bf FEniCS Demos.} These documented demonstration programs are
  a great way to learn the different features in FEniCS. They can be
  found at:\\
%
\url{http://fenicsproject.org/documentation/dolfin/1.5.0/python/demo/index.html}\\
%
They come already packaged in FEniCS when you install it; the
appropriate directory depends on your operating system, and the paths
can be found here:\\
%
\url{http://fenicsproject.org/documentation/demos.html#finding-demos}\\
%

\item {\bf Quick Programmer's References.} Some of the classes and
  functions in DOLFIN are more frequently used than others. The Python
  implementations are described in
%
\url{http://fenicsproject.org/documentation/dolfin/1.5.0/python/quick_reference.html}. See also
\url{http://fenicsproject.org/documentation/dolfin/1.5.0/python/genindex.html}
%
for the Complete Programmer's References.
%

\item {\bf Getting Help.} See: \url{http://fenicsproject.org/support/}

\eit

Other resources, although a little outdated and not fully compatible with the latest versions of FEniCS, include 
\bit
\item {\bf FEniCS Tutorial:} This is the best starting point; it
  describes the Python interface to FEniCS:\\
%\centerline{\url{http://fenicsproject.org/documentation/tutorial/index.html}}
\url{http://fenicsproject.org/documentation/tutorial/index.html}\\
The tutorial is also available as a PDF document:\\
\url{http://fenicsproject.org/_static/tutorial/fenics_tutorial_1.0.pdf}\\
All of the Python codes described in the tutorial can be downloaded as
a tarball from:\\
\url{http://fenicsproject.org/_static/tutorial/fenics_tutorial_examples.tar.gz}\\
%In particular, the Poisson equation example we'll be discussing in
%class is available as the code {\tt d1\_p2D.py} in the directory {\tt
%  stationary/poisson} once you 
%untar the compressed file above. [{\bf Note that this tutorial is severely outdated} and several modifications are required run these examplers with FEniCS 1.5.0. For example, the function {\tt
%    UnitSquare} for generating a mesh has been replaced by {\tt
%    UnitSquareMesh}.] 

\item {\bf The FEniCS Book:} All 732 pages of the FEniCS book ({\em
  Automated Solution of Differential Equations by the Finite Element
  Method}) can be downloaded (legally!) from here:\\
%  \url{https://launchpadlibrarian.net/83776282/fenics-book-2011-10-27-final.pdf}\\
\url{http://launchpad.net/fenics-book/trunk/final/+download/fenics-book-2011-10-27-final.pdf}\\
%
This is the comprehensive reference to FEniCS, along with many
examples of the applications of FEniCS to problems in science and
engineering. You will notice that the first chapter of the book is
just the FEniCS Tutorial (with some minor editorial differences).

\item {\bf The FEniCS Manual.} This is a 200-page excerpt from the
  FEniCS Book, including the FEniCS Tutorial, an introduction to the
  finite element method, and documentation of DOLFIN and UFL:\\
%
\url{http://launchpad.net/fenics-book/trunk/final/+download/fenics-manual-2011-10-31.pdf}\\
%
Since it's an excerpt from the FEniCS Book, you probably won't need
it. 


\eit

\section{Installing FEniCS}

All of the Python codes for the examples and supporting materials for
the exercises in this course require FEniCS version 1.5.0.  Previous
versions of FEniCS are not supported, and minor modifications are
required to run the materials using FEniCS 1.6.0.
%
The following steps will get you up and running with FEniCS on your system.
\bit

\item {\bf MacOS 10.9 and 10.10 systems:}\footnote{FEniCS does not fully support MacOS 10.11 (El Capitan).
If you are using MacOS 10.11, please follow the instructions below under
\emph{Any other OS}. }\\ 
Download FEniCS 1.5.0 from  \url{http://fenicsproject.org/download/older_releases.html#older-releases}.
  Find your MacOS version (either 10.9 or 10.10) and download the appropriate binaries of FEniCS 1.5.0.
  If you are running {\tt bash} as default shell, you can add the following line to your {\tt .profile} file in your
  home directory:\\
%
{\tt source
  /Applications/FEniCS.app/Contents/Resources/share/fenics/fenics.conf}\\
%
Alternatively you can just double-click on the FEniCS icon in your
Applications directory and that will generate a new shell
preconfigured with the paths that FEniCS needs. Just run FEniCS from
within this shell. 

FEniCS demo programs are located under \\
{\tt /Applications/FEniCS.app/Contents/Resources/share/dolfin/demo }

\item {\bf Ubuntu LTS 14.04:}\footnote{The same installation
  instructions may also work for different version of Ubuntu or other
  Debian-based versions of Linux.}\\ 
Open a shell and run the following commands:
\begin{verbatim}
sudo add-apt-repository ppa:fenics-packages/fenics-1.5.x
sudo apt-get update 
sudo apt-get install -y fenics
sudo apt-get dist-upgrade
sudo apt-get install -y ipython-notebook
sudo apt-get install -y paraview
\end{verbatim}
{\bf NOTE: to minimize conflicts with other libraries already installed on your system it is suggested to start from a fresh installation of Ubuntu.}\\
If in the future you decide to uninstall FEniCS and remove all its
dependencies, you can run the following commands:
\begin{verbatim}
sudo apt-get purge --auto-remove fenics
sudo ppa-purge ppa:fenics-packages/fenics-1.5.x
\end{verbatim}

FEniCS demo programs are located under \\
{\tt /usr/share/dolfin/demo/ }.\\

\item {\bf ICES Linux Desktop:}\\
Open a shell and run the following commands:
\begin{verbatim}
module purge
module load sl6 gcc/4.8 python cmake mpich2/1.4.1p1 mkl vtk petsc fenics
\end{verbatim}
{\bf Note: you will have to run the above commands each time you open
  a new shell.} If you would like to run FEniCS without having to type
these commands each time you open a new shell, you can add the
commands in your default shell profile (e.g.\ {\tt .bashrc} if your
default shell is bash).\\ 
{\bf WARNING: The ICES Linux Desktop does not support ipython notebooks at this time. We have sent a support request to Sysnet to address this issue.}\\
FEniCS demo programs are located under \\
{\tt /usr/share/dolfin/demo/ }\\
You should copy this folder to your home directory if you wish to run any of the examples.\\

\item {\bf Any other OSes, including Windows, MacOS 10.11, and other
  Linux distributions:} \\
You can download a Linux virtual machine running Ubuntu 14.04 and
FEniCS 1.5 from \\
\url{http://users.ices.utexas.edu/~uvilla/hippylib.ova} \\
(Warning: this file is 3.1GB!)
To run the virtual machine you should use VirtualBox from here:\\
\url{https://www.virtualbox.org/wiki/Downloads} \\
Instructions on how to import and run the virtual machine can be found
at:\\
\url{https://docs.oracle.com/cd/E26217_01/E26796/html/qs-import-vm.html} \\
The username and password for the virtual machine are both {\tt hippylib}.

\eit

\noindent To check the FEniCS installation on your system, go to the
          {\tt demo} folder (see above for the exact location of this
          folder in your system) and run {\tt demo\_poisson.py} by
          typing the following at the command lines:\\
%
\\
{\tt cd documented/poisson/python/}\\
{\tt python  demo\_poisson.py}\\
\\
%
After compilation, you'll get a message saying \\
%
\\
{\tt Solving linear variational problem.}\\
\\
%
and a simple plot of the solution will pop up in a new window. 
If you mouse over the {\tt Help} tag in the low left, you'll get a list of some keyboard
commands for interactive visualization. You can quit this run by
typing {\tt q} or {\tt Q}. 

The Poisson equation code {\tt poisson\_demo.py} is described
  extensively in the FEniCS Tutorial (pages 5--15), as well as online
  at\\
\url{
  http://fenicsproject.org/documentation/dolfin/1.5.0/python/demo/documented/poisson/python/documentation.html}\\

\section{Additional help}
If you have questions regarding FEniCS or have problems running with the FEniCS examples provided in this class, please feel free to contact Umberto Villa (\url{uvilla@ices.utexas.edu}).

\end{document}

